% -*- mode: noweb; noweb-default-code-mode: R-mode; -*-
%\VignetteIndexEntry{Design decisions and implementation}

\documentclass[a4paper,10pt]{amsart}

\usepackage{ucs}
\usepackage[utf8x]{inputenc}
\usepackage[T1]{fontenc}
\usepackage[sort&compress]{natbib}
\usepackage{hyperref}
\usepackage{graphics}
\usepackage{sidecap}
\setlength{\captionindent}{0pt}
\usepackage{url}
\renewcommand{\floatpagefraction}{0.8}

\author{Jari Oksanen}
\title{Design decisions and implementation details in vegan}
\date{$ $Id: decision-vegan.Rnw 1592 2011-04-27 11:30:32Z jarioksa $ $
  processed with vegan
1.17-12
in R version 2.13.0 (2011-04-13) on \today}

\usepackage{Sweave}
\begin{document}

\setkeys{Gin}{width=0.55\linewidth}

\maketitle

\begin{abstract}

\noindent This document describes design decisions, and discusses implementation
and algorithmic details in some vegan functions. The proper FAQ is
another document.

\end{abstract}

\tableofcontents

\section{Nestedness and Null models}

Some indicators of nestedness and null models of communities are only
described in general terms, and they could be implemented in various
ways. Here I discuss the implementation in \texttt{vegan}.

\subsection{Matrix temperature}

The matrix temperature is intuitively simple
(Fig. \ref{fig:nestedtemp}), but the the exact calculations were not
explained in the original publication \cite{AtmarPat93}.
\begin{SCfigure}
\includegraphics{decision-vegan-002}
\label{fig:nestedtemp}
\caption{Matrix temperature for \emph{Falco subbuteo} on Sibbo
  Svartholmen (dot). The curve is the fill line, and in a cold
  matrix, all presences (red squares) should be in the upper left
  corner behind the fill line. Dashed diagonal line of length $D$ goes
  through the point, and an arrow of length $d$ connects the point to
  the fill line. The ``surprise'' for this point is $u = (d/D)^2$ and
  the matrix temperature is based on the sum of surprises: presences
  outside the fill line or absences within the fill line.}
\end{SCfigure}
The function can be implemented in many ways following the general
principles.  Rodr{\'i}guez-Giron{\'e}s and Santamaria \cite{RodGir06}
have seen the original code and reveal more details of calculations,
and their explanation is the basis of the implementation in
\texttt{vegan}.  However, there are still some open issues, and
probably \texttt{vegan} function \texttt{nestedtemp} will never
exactly reproduce results from other programs, although it is based on
the same general principles. I try to give main computation details in
this documents --- all details can be seen in the source code of
\texttt{nestedtemp}.

\begin{itemize}
\item Species and sites are put into unit square \citep{RodGir06}. The
  coordinates for $n$ item will be $(k-0.5)/n$ for $k=1 \ldots n$, so
  that there are no points in the corners or the margins of the unit
  square, and a diagonal line can be drawn through any point. I do not
  know how the rows and columns are converted to the unit square in
  other software, and this may be a considerable source of differences
  among implementations.
  \item Species and sites are ordered alternately using indices
    \citep{RodGir06}:
    \begin{equation}
    \begin{split}
      s_j &= \sum_{i|x_{ij} = 1} i^2 \\
      t_j &= \sum_{i|x_{ij} = 0} (n-i+1)^2 
    \end{split}
    \end{equation}
    Here $x$ is the data matrix, where $1$ is presence, and $0$ is
    absence, $i$ and $j$ are row and column indices, and $n$ is the
    number of rows. The equations give the indices for columns, but
    the indices can be reversed for corresponding row indexing.
    Ordering by $s$ packs presences to the top left corner, and
    ordering by $t$ pack zeros away from the top left corner. The final
    sorting should be ``a compromise'' \cite{RodGir06} between these
    scores, and \texttt{vegan} uses $s+t$.  The result should be cool,
    but the packing does not try to minimize the temperature
    \citep{RodGir06}.  I do not know how the ``compromise'' is
    defined, and this can cause some differences to other
    implementations.
  \item The following function is used to define the fill line:
    \begin{equation}
      y = (1-(1-x)^p)^{1/p}
    \end{equation}
    This is similar to the equation suggested by
    \cite[eq. 4]{RodGir06}, but omits all terms dependent on the
    numbers of species or sites, because I could not understand why
    they were needed. The differences are visible only in small data
    sets. The $y$ and $x$ are the coordinates in the unit square, and
    the parameter $p$ is selected so that the curve covers the same
    area as is the proportion of presences
    (Fig. \ref{fig:nestedtemp}). The parameter $p$ is found
    numerically using \textsf{R} functions \texttt{integrate} and
    \texttt{uniroot}.  The fill line used in the original matrix
    temperature software \cite{AtmarPat93} is supposed to be similar
    \citep{RodGir06}. Small details in the fill line combined with
    differences in scores used in the unit square (especially in the
    corners) can cause large differences in the results.
  \item A line with slope $-1$ is drawn through the point and the $x$
    coordinate of the intersection of this line and the fill line is
    found using function \texttt{uniroot}. The difference of this
    intersection and the row coordinate gives the argument $d$ of matrix
    temperature (Fig. \ref{fig:nestedtemp}).
  \item In other software, ``duplicated'' species occurring on every
    site are removed, as well as empty sites and species after
    reordering \cite{RodGir06}. This is not done in \texttt{vegan}.
\end{itemize}

\subsection{Backtracking}

Gotelli's and Entsminger's seminal paper \cite{GotelliEnt01} on filling
algorithms is somewhat confusing: it explicitly deals with ``knight's
tour'' which is quite a different problem than the one we face with
null models.  The chess piece ``knight''\footnote{``Knight'' is
  ``Springer'' in German which is very appropriate as Springer was the
  publisher of the paper on ``knight's tour''} has a history:
a piece in a certain position could only have entered from some
candidate squares. The filling of incidence matrix has not such a history:
if we know that the item last added was in certain row and column, we
have no information to guess which of the filled items was entered
previously. A consequence of dealing with a different problem is that
\cite{GotelliEnt01} does not give many hints on implementing a fill
algorithm as a community null model.

The backtracking is implemented in two stages in \textbf{vegan}: filling and
backtracking.
\begin{enumerate}
  \item The matrix is filled in the order given by the marginal
    probabilities. In this way the matrix will look similar to the
    final matrix at all stages of filling. Equal filling probabilities
    are not used since that is ineffective and produces strange fill
    patterns: the rows and columns with one or a couple of presences
    are filled first, and the process is cornered to columns and
    rows with many presences. As a consequence, the the process tries
    harder to fill that corner, and the result is a more tightly
    packed quadratic fill pattern than with other methods.
  \item The filling stage stops when no new points can be added
    without exceeding row or column totals. ``Backtracking'' means
    removing random points and seeing if this allows adding new points
    to the plot. No record of history is kept (and there is no reason
    to keep a record of history), but random points are removed and
    filled again. The number of removed points increases from one to
    four points. New configuration is kept if it is at least as good
    as the previous one, and the number of removed points is reduced
    back to one if the new configuration is better than the old one.
    Because there is no record of history, this does not sound like a
    backtracking, but it still fits the general definition of
    backtracking: ``try something, and if it fails, try something
    else'' \cite{Sedgewick90}.
\end{enumerate}


\section{Scaling in redundancy analysis}

This chapter discusses the scaling of scores (results) in redundancy
analysis and principal component analysis performed by function
\texttt{rda} in the \texttt{vegan} library.  

Principal component analysis, and hence redundancy analysis, is a case
of singular value decomposition (\textsc{svd}).  Functions
\texttt{rda} and \texttt{prcomp} even use \textsc{svd} internally in
their algorithm.

In \textsc{svd} a centred data matrix is decomposed into orthogonal
components so that $x_{ij} = \sum_k \sigma_k u_{ik} v_{jk}$, where
$u_{ik}$ and $v_{jk}$ are orthonormal coefficient matrices and
$\sigma_k$ are singular values.  Orthonormality means that sums of
squared columns is one and their cross-product is zero, or $\sum_i
u_{ik}^2 = \sum_j v_{jk}^2 = 1$, and $\sum_i u_{ik} u_{il} = \sum_j
v_{jk} v_{jl} = 0$ for $k \neq l$. This is a decomposition, and the
original matrix is found exactly from the singular vectors and
corresponding singular values, and first two singular components give
the rank $=2$ least squares estimate of the original matrix.

Principal component analysis is often presented (and performed in
legacy software) as an eigenanalysis of covariance matrices.  Instead
of a data matrix, we analyse a matrix of covariances and variances
$\mathbf{S}$.  The result are orthonormal coefficient matrix
$\mathbf{U}$ and eigenvalues $\mathbf{\Lambda}$.  The coefficients
$u_{ik}$ ares identical to \textsc{svd} (except for possible sign
changes), and eigenvalues $\lambda_k$ are related to the corresponding
singular values by $\lambda_k = \sigma_k^2 /(n-1)$.  With classical
definitions, the sum of all eigenvalues equals the sum of variances of
species, or $\sum_k \lambda_k = \sum_j s_j^2$, and it is often said
that first axes explain a certain proportion of total variance in the
data.  The orthonormal matrix $\mathbf{V}$ of \textsc{svd} can be
found indirectly as well, so that we have the same components in both
methods.

The coefficients $u_{ik}$ and $v_{jk}$ are scaled similarly for all
axes $k$. Singular values $\sigma_k$ or eigenvalues $\lambda_k$ give
the information of the importance of axes, or the `axis lengths.'
Instead of the orthonormal coefficients, or equal length axes, it is
customary to scale species (column) or site (row) scores or both by
eigenvalues to display the importance of axes and to describe the true
configuration of points.  Table \ref{tab:scales} shows some
alternative scalings.  These alternatives apply to principal
components analysis in all cases, and in redundancy analysis, they
apply to species scores and constraints or linear combination scores;
weighted averaging scores have somewhat wider dispersion.

\begin{table}
  \caption{\label{tab:scales} Alternative scalings for \textsc{rda} used
    in the functions \texttt{prcomp} and \texttt{princomp}, and the
    one used in the \texttt{vegan} function \texttt{rda} 
    and the proprietary software \texttt{Canoco}
    scores in terms of orthonormal species ($u_{ik}$) and site scores
    ($v_{jk}$), eigenvalues ($\lambda_k$), number of sites  ($n$) and
    species standard deviations ($s_j$). In \texttt{rda},
    $\mathrm{const} = \sqrt[4]{(n-1) \sum \lambda_k}$.  Corresponding
    negative scaling in \texttt{vegan}
   % and corresponding positive scaling in \texttt{Canoco 3}  
    is derived
    dividing each  species by its standard deviation $s_j$ (possibly
    with some additional constant multiplier).  }
\begin{tabular}{lcc}
& \textbf{Site scores} $u_{ik}^*$ &
\textbf{Species scores} $v_{jk}^*$ \\
\texttt{prcomp, princomp} &
$u_{ik} \sqrt{n-1} \sqrt{\lambda_k}$ &
$v_{jk}$ \\
\texttt{rda, scaling=1} &
$u_{ik} \sqrt{\lambda_k/ \sum \lambda_k} \times \mathrm{const}$ &
$v_{jk} \times \mathrm{const}$
\\
\texttt{rda, scaling=2} &
$u_{ik} \times \mathrm{const}$ &
$v_{jk} \sqrt{\lambda_k/ \sum \lambda_k} \times \mathrm{const}$  \\
\texttt{rda, scaling=3} &
$u_{ik} \sqrt[4]{\lambda_k/ \sum \lambda_k} \times \mathrm{const}$ &
$v_{jk} \sqrt[4]{\lambda_k/ \sum \lambda_k} \times \mathrm{const}$ \\
\texttt{rda, scaling < 0} &
$u_{ik}^*$ &
$\sqrt{\sum \lambda_k /(n-1)} s_j^{-1} v_{jk}^*$
% \\
% \texttt{Canoco 3, scaling=-1} &
% $u_{ik} \sqrt{n-1} \sqrt{\lambda_k / \sum \lambda_k}$ &
% $v_{jk} \sqrt{n}$ \\
% \texttt{Canoco 3, scaling=-2} &
% $u_{ik} \sqrt{n-1}$ &
% $v_{jk} \sqrt{n} \sqrt{\lambda_k / \sum \lambda_k}$
% \\
% \texttt{Canoco 3, scaling=-3} &
% $u_{ik} \sqrt{n-1} \sqrt[4]{\lambda_k / \sum \lambda_k}$ &
% $v_{jk} \sqrt{n} \sqrt[4]{\lambda_k / \sum \lambda_k}$
\end{tabular}
\end{table}

In community ecology, it is common to plot both species and sites in
the same graph.  If this graph is a graphical display of \textsc{svd},
or a graphical, low-dimensional approximation of the data, the graph
is called a biplot.  The graph is a biplot if the transformed scores
satisfy $x_{ij} = c \sum_k u_{ij}^* v_{jk}^*$ where $c$ is a scaling
constant.  In functions \texttt{princomp}, \texttt{prcomp} and
\texttt{rda}, $c=1$ and the plotted scores are a biplot so that the
singular values (or eigenvalues) are expressed for sites, and species
are left unscaled.  
% For \texttt{Canoco 3} $c = n^{-1} \sqrt{n-1}
% \sqrt{\sum \lambda_k}$ with negative \texttt{Canoco} scaling
% values. All these $c$ are constants for a matrix, so these are all
% biplots with different internal scaling of species and site scores
% with respect to each other.  For \texttt{Canoco} with positive scaling
% values and \texttt{vegan} with negative scaling values, no constant
% $c$ can be found, but the correction is dependent on species standard
% deviations $s_j$, and these scores do not define a biplot.

There is no natural way of scaling species and site scores to each
other.  The eigenvalues in redundancy and principal components
analysis are scale-dependent and change when the the data are
multiplied by a constant.  If we have percent cover data, the
eigenvalues are typically very high, and the scores scaled by
eigenvalues will have much wider dispersion than the orthonormal set.
If we express the percentages as proportions, and divide the matrix by
$100$, the eigenvalues will be reduced by factor $100^2$, and the
scores scaled by eigenvalues will have a narrower dispersion.  For
graphical biplots we should be able to fix the relations of row and
column scores to be invariant against scaling of data.  The solution
in R standard function \texttt{biplot} is to scale site and species
scores independently, and typically very differently, but plot each
independently to fill the graph area.  The solution in \texttt{Canoco} and 
and \texttt{rda} is to use proportional eigenvalues $\lambda_k / \sum
\lambda_k$ instead of original eigenvalues.  These proportions are
invariant with scale changes, and typically they have a nice range for
plotting two data sets in the same graph.

The \textbf{vegan} package uses a scaling constant $c = \sqrt[4]{(n-1)
  \sum \lambda_k}$ in order to be able to use scaling by proportional
eigenvalues (like in \texttt{Canoco}) and still be able to have a
biplot scaling. Because of this, the scaling of \texttt{rda} scores is
non-standard. However, the \texttt{scores} function lets you to set
the scaling constant to any desired values. It is also possible to
have two separate scaling constants: the first for the species, and
the second for sites and friends, and this allows getting scores of
other software or R functions (Table \ref{tab:rdaconst}). 
\begin{table}
  \caption{\label{tab:rdaconst} Values of the \texttt{const} argument in
    \textbf{vegan} to get the scores that are equal to those from
    other functions and software. Number of sites (rows) is $n$, 
    the number of species (columns) is $m$, and the sum of all
    eigenvalues is $\sum_k \lambda_k$ (this is saved as the item
    \texttt{tot.chi} in the \texttt{rda} result)}.
\begin{tabular}{lccc}
& \textbf{Scaling} &\textbf{Species constant} & \textbf{Site constant} \\
\texttt{vegan} & any  & $\sqrt[4]{(n-1) \sum \lambda_k}$ & $\sqrt[4]{(n-1) \sum \lambda_k}$\\
\texttt{prcomp}, \texttt{princomp} & \texttt{1} & $1$ & $\sqrt{(n-1) \sum_k \lambda_k}$\\
\texttt{Canoco 3} & \texttt{-1, -2, -3} & $\sqrt{n-1}$ & $\sqrt{n}$\\
\texttt{Canoco 4} & \texttt{-1, -2, -3} & $\sqrt{m}$ & $\sqrt{n}$
\end{tabular}
\end{table}

In this chapter, I used always centred data matrices.  In principle
\textsc{svd} could be done with original, non-centred data, but
there is no option for this in \texttt{rda}, because I think that
non-centred analysis is dubious and I do not want to encourage its use
(if you think you need it, you are certainly so good in programming
that you can change that one line in \texttt{rda.default}).  I do
think that the arguments for non-centred analysis are often twisted,
and the method is not very good for its intended purpose, but there
are better methods for finding fuzzy classes.  Normal, centred
analysis moves the origin to the average of all species, and the
dimensions describe differences from this average.  Non-centred
analysis leaves the origin in the empty site with no species, and the
first axis usually runs from the empty site to the average
site. Second and third non-centred components are often very similar
to first and second (etc.) centred components, and the best way to use
non-centred analysis is to discard the first component and use only
the rest. This is better done with directly centred analysis.


\section{Why to use weighted averages scores instead of linear
  combinations in constrained ordination}

Constrained ordination methods such as Constrained Correspondence
Analysis (CCA) and Redundancy Analysis (RDA) produce two kind of site
scores \cite{Braak86, Palmer93}:
\begin{itemize}
\item
LC or Linear Combination Scores which are linear combinations of
constraining variables.
\item
WA or Weighted Averages Scores which are such weighted averages of
species scores that are as similar to LC scores as possible.
\end{itemize}
Many computer programs for constrained ordinations give only or
primarily LC scores, following Mike Palmer's recommendation
\cite{Palmer93}.  However, functions \texttt{cca} and \texttt{rda} in
the \texttt{vegan} package use primarily WA scores. This chapter
explains the reasons for this choice.

Briefly, the main reasons are that
\begin{itemize}
\item
LC scores \emph{are} linear combinations, so they give us only the
(scaled) environmental variables. This means that they are
independent of vegetation and cannot be found from the species
composition.  Moreover, identical combinations of environmental
variables give identical LC scores irrespective of vegetation.
\item
Bruce McCune has demonstrated that noisy environmental variables
result in deteriorated LC scores whereas WA scores tolerate some errors
in environmental variables \cite{McCune97}.  All environmental
measurements contain some errors, and therefore it is safer to use WA
scores.
\end{itemize}
This article studies mainly the first point.  The users of
\texttt{vegan} have a choice of either LC or WA (default) scores, but
after reading this article, I believe that most of them do not want to
use LC scores, because they are not what they were looking for in
ordination.

\subsection{LC Scores are Linear Combinations}

Let us perform a simple CCA analysis using only two environmental
variables so that we can see the constrained solution completely in
two dimensions:
\begin{Schunk}
\begin{Sinput}
> library(vegan)
> data(varespec)
> data(varechem)
> orig <- cca(varespec ~ Al + K, varechem)
\end{Sinput}
\end{Schunk}
Function \texttt{cca} in \texttt{vegan} uses WA scores as
default. So we must specifically ask for LC scores
(Fig. \ref{fig:ccalc}).
\begin{Schunk}
\begin{Sinput}
> plot(orig, dis = c("lc", "bp"))
\end{Sinput}
\end{Schunk}
\begin{SCfigure}
\includegraphics{decision-vegan-005}
\caption{LC scores in CCA of the original data.}
\label{fig:ccalc}
\end{SCfigure}

What would happen to linear combinations of LC scores if we shuffle
the ordering of sites in species data?  Function \texttt{sample()} below
shuffles the indices.
\begin{Schunk}
\begin{Sinput}
> i <- sample(nrow(varespec))
> shuff <- cca(varespec[i, ] ~ Al + K, varechem)
\end{Sinput}
\end{Schunk}
\begin{SCfigure}
\includegraphics{decision-vegan-007}
\caption{LC scores of shuffled species data.}
\label{fig:ccashuff}
\end{SCfigure}
It seems that site scores are fairly similar, but oriented differently
(Fig. \ref{fig:ccashuff}).  We can use Procrustes rotation to see how
similar the site scores indeed are (Fig. \ref{fig:ccaproc}).
\begin{Schunk}
\begin{Sinput}
> plot(procrustes(scores(orig, dis = "lc"), scores(shuff, dis = "lc")))
\end{Sinput}
\end{Schunk}
\begin{SCfigure}
\includegraphics{decision-vegan-009}
\caption{Procrustes rotation of LC scores from CCA of original and shuffled data.}
\label{fig:ccaproc}
\end{SCfigure}
There is a small difference, but this will disappear if we use
Redundancy Analysis (RDA) instead of CCA
(Fig. \ref{fig:rdaproc}). Here we use a new shuffling as well.
\begin{Schunk}
\begin{Sinput}
> tmp1 <- rda(varespec ~ Al + K, varechem)
> i <- sample(nrow(varespec))
> tmp2 <- rda(varespec[i, ] ~ Al + K, varechem)
\end{Sinput}
\end{Schunk}
\begin{SCfigure}
\includegraphics{decision-vegan-011}
\caption{Procrustes rotation of LC scores in RDA of the original and shuffled data.}
\label{fig:rdaproc}
\end{SCfigure}

LC scores indeed are linear combinations of constraints (environmental
variables) and \emph{independent of species data}: You can
shuffle your species data, or change the data completely, but the LC
scores will be unchanged in RDA.  In CCA the LC scores are
\emph{weighted} linear combinations with site totals of species data
as weights. Shuffling species data in CCA changes the weights, and
this can cause changes in LC scores.  The magnitude of changes depends
on the variability of site totals.

The original data and shuffled data differ in their goodness of
fit\footnote{Or probably differ: The randomization is done while
generating this article, and different versions may have different
randomizations.}.
\begin{Schunk}
\begin{Sinput}
> orig
\end{Sinput}
\begin{Soutput}
Call: cca(formula = varespec ~ Al + K, data = varechem)

              Inertia Proportion Rank
Total          2.0832     1.0000     
Constrained    0.4760     0.2285    2
Unconstrained  1.6072     0.7715   21
Inertia is mean squared contingency coefficient 

Eigenvalues for constrained axes:
  CCA1   CCA2 
0.3608 0.1152 

Eigenvalues for unconstrained axes:
    CA1     CA2     CA3     CA4     CA5     CA6     CA7     CA8 
0.37476 0.24036 0.19696 0.17818 0.15209 0.11840 0.08364 0.07567 
(Showed only 8 of all 21 unconstrained eigenvalues)
\end{Soutput}
\begin{Sinput}
> shuff
\end{Sinput}
\begin{Soutput}
Call: cca(formula = varespec[i, ] ~ Al + K, data = varechem)

              Inertia Proportion Rank
Total         2.08320    1.00000     
Constrained   0.16985    0.08154    2
Unconstrained 1.91334    0.91846   21
Inertia is mean squared contingency coefficient 

Eigenvalues for constrained axes:
   CCA1    CCA2 
0.12003 0.04983 

Eigenvalues for unconstrained axes:
    CA1     CA2     CA3     CA4     CA5     CA6     CA7     CA8 
0.45421 0.33357 0.23089 0.19059 0.15665 0.11689 0.10654 0.08501 
(Showed only 8 of all 21 unconstrained eigenvalues)
\end{Soutput}
\end{Schunk}
Similarly their WA scores will be (probably) very different
(Fig. \ref{fig:ccawa}).
\begin{SCfigure}
\includegraphics{decision-vegan-013}
\caption{Procrustes rotation of WA scores of CCA with the original and
  shuffled data.}
\label{fig:ccawa}
\end{SCfigure}

The example used only two environmental variables so that we can
easily plot all constrained axes.  With a larger number of
environmental variables the full configuration remains similarly
unchanged, but its orientation may change, so that two-dimensional
projections look different.  In the full space, the differences should
remain within numerical accuracy:
\begin{Schunk}
\begin{Sinput}
> tmp1 <- rda(varespec ~ ., varechem)
> tmp2 <- rda(varespec[i, ] ~ ., varechem)
> proc <- procrustes(scores(tmp1, dis = "lc", choi = 1:14), scores(tmp2, 
+     dis = "lc", choi = 1:14))
> max(residuals(proc))
\end{Sinput}
\begin{Soutput}
[1] 2.516303e-14
\end{Soutput}
\end{Schunk}
In \texttt{cca} the difference would be somewhat larger than now
observed 2.5163e-14 because site
weights used for environmental variables are shuffled with the species
data.

\subsection{Factor constraints}

It seems that users often get confused when they perform constrained
analysis using  only one factor (class variable) as constraint.  The
following example uses the classical dune meadow data \cite{Jongman87}:
\begin{Schunk}
\begin{Sinput}
> data(dune)
> data(dune.env)
> summary(dune.env)
\end{Sinput}
\begin{Soutput}
       A1         Moisture Management       Use    Manure
 Min.   : 2.800   1:7      BF:3       Hayfield:7   0:6   
 1st Qu.: 3.500   2:4      HF:5       Haypastu:8   1:3   
 Median : 4.200   4:2      NM:6       Pasture :5   2:4   
 Mean   : 4.850   5:7      SF:6                    3:4   
 3rd Qu.: 5.725                                    4:3   
 Max.   :11.500                                          
\end{Soutput}
\begin{Sinput}
> orig <- cca(dune ~ Moisture, dune.env)
\end{Sinput}
\end{Schunk}
When the results are plotted using LC scores, sample plots fall only
in four alternative positions (Fig. \ref{fig:factorlc}).
\begin{SCfigure}
\includegraphics{decision-vegan-016}
\caption{LC scores of the dune meadow data using only one factor as a
  constraint.}
\label{fig:factorlc}
\end{SCfigure}
In the previous chapter we saw that this happens because LC scores
\emph{are} the environmental variables, and they can be distinct only
if the environmental variables are distinct.  However, normally the user
would like to see how well the environmental variables separate the
vegetation, or inversely, how we could use the vegetation to
discriminate the environmental conditions.  For this purpose we should
plot WA scores, or LC scores and WA scores together:  The LC scores
show where the site \emph{should} be, the WA scores shows where the
site \emph{is}.

Function \texttt{ordispider} adds line segments to connect each WA
score with the corresponding LC (Fig.  \ref{fig:walcspider}).
\begin{Schunk}
\begin{Sinput}
> plot(orig, display = "wa", type = "points")
> ordispider(orig, col = "red")
> text(orig, dis = "cn", col = "blue")
\end{Sinput}
\end{Schunk}
\begin{SCfigure}
\includegraphics{decision-vegan-018}
\caption{A ``spider plot'' connecting WA scores to corresponding LC
  scores. The shorter the web segments, the better the ordination.}
\label{fig:walcspider}
\end{SCfigure}
This is the standard way of displaying results of discriminant
analysis, too.  Moisture classes \texttt{1} and \texttt{2} seem to be
overlapping, and cannot be completely separated by their
vegetation. Other classes are more distinct, but there seems to be a
clear arc effect or a ``horseshoe'' despite using CCA.

\subsection{Conclusion}

LC scores are only the (weighted and scaled) constraints and
independent of vegetation. If you plot them, you plot only your
environmental variables. WA scores are based on vegetation data but
are constrained to be as similar to the LC scores as only
possible. Therefore \texttt{vegan} calls LC scores as
\texttt{constraints} and WA scores as \texttt{site scores}, and uses
primarily WA scores in plotting.  However, the user makes the ultimate
choice, since both scores are available.


\bibliographystyle{plain}
\bibliography{vegan}

\end{document}
