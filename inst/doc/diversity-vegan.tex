% -*- mode: noweb; noweb-default-code-mode: R-mode; -*-
%\VignetteIndexEntry{Diversity analysis in vegan}
\documentclass[a4paper,10pt]{amsart}
\usepackage{ucs}
\usepackage[utf8x]{inputenc}
\usepackage[T1]{fontenc}
\usepackage{graphicx}
\usepackage{sidecap}
\setlength{\captionindent}{0pt}
\usepackage{url}


\title{Vegan: ecological diversity}
\author{Jari Oksanen}
\date{$ $Id: diversity-vegan.Rnw 1036 2009-10-07 07:06:36Z jarioksa $ $
  processed with vegan 1.17-0
  in R version 2.10.1 (2009-12-14) on \today}
\usepackage{Sweave}
\begin{document}
\setkeys{Gin}{width=0.55\linewidth}

\maketitle

\tableofcontents

\noindent The \texttt{vegan} package has two major components:
multivariate analysis (mainly ordination), and methods for diversity
analysis of ecological communities.  This document gives an
introduction to the latter.  Ordination methods are covered in other
documents.  Many of the diversity functions were written by Roeland
Kindt and Bob O'Hara.

Most diversity methods assume that data are counts of individuals.
The methods are used with other data types, and some people argue that
biomass or cover are more adequate than counts of individuals of
variable sizes.  However, this document mainly uses a data set with
counts: stem counts of trees on $1$ha plots in the Barro Colorado
Island.  The following steps make these data available for the
document:
\begin{Schunk}
\begin{Sinput}
> library(vegan)
> data(BCI)
\end{Sinput}
\end{Schunk}

\section{Diversity indices}

Function \texttt{diversity} finds the most commonly used diversity
indices:
\begin{align}
H &= - \sum_{i=1}^S p_i \log_b  p_i & \text{Shannon--Weaver}\\
D_1 &= 1 - \sum_{i=1}^S p_i^2  &\text{Simpson}\\
D_2 &= \frac{1}{\sum_{i=1}^S p_i^2}  &\text{inverse Simpson}
\end{align}
where $p_i$ is the proportion of species $i$, and $S$ is the number of
species so that $\sum_{i=1}^S p_i = 1$, and $b$ is the base of the
logarithm.  It is most common to use natural logarithms (and then we
mark index as $H'$), but $b=2$ has
theoretical justification. The default is to use natural logarithms.
Shannon index is calculated with:
\begin{Schunk}
\begin{Sinput}
> H <- diversity(BCI)
\end{Sinput}
\end{Schunk}
which finds diversity indices for all sites.

\texttt{Vegan} does not have indices for evenness (equitability), but
the most common of these, Pielou's evenness $J = H'/\log(S)$ is easily
found as:
\begin{Schunk}
\begin{Sinput}
> J <- H/log(specnumber(BCI))
\end{Sinput}
\end{Schunk}
where \texttt{specnumber} is a simple \texttt{vegan} function to find
the numbers of species.

\texttt{Vegan} also can estimate R\'{e}nyi diversities of order $a$:
\begin{equation}
H_a = \frac{1}{1-a} \log \sum_{i=1}^S p_i^a
\end{equation}
or the corresponding Hill numbers $N_a = \exp(H_a)$.  Many common
diversity indices are special cases of Hill numbers: $N_0 = S$, $N_1 =
\exp(H')$, $N_2 = D_2$, and $N_\infty = 1/(\max p_i)$. The
corresponding R\'{e}nyi diversities are $H_0 = \log(S)$, $H_1 = H'$, $H_2 =
- \log(\sum p_i^2)$, and $H_\infty = - \log(\max p_i)$.  We select a
random subset of five sites for R\'{e}nyi diversities:
\begin{Schunk}
\begin{Sinput}
> k <- sample(nrow(BCI), 6)
> R <- renyi(BCI[k, ])
\end{Sinput}
\end{Schunk}
We can really regard a site more diverse if all of its R\'{e}nyi
diversities are higher than in another site.  We can inspect this
graphically using the standard \texttt{plot} function for the
\texttt{renyi} result (Fig. \ref{fig:renyi}).
\begin{SCfigure}
\includegraphics{diversity-vegan-007}
\caption{R\'{e}nyi diversities in six randomly selected plots. The plot
  uses Trellis graphics with a separate panel for each site. The dots
  show the values for sites, and the lines the extremes and median in
  the data set.}
\label{fig:renyi}
\end{SCfigure}

Finally, the $\alpha$ parameter of Fisher's log-series can be used as
a diversity index:
\begin{Schunk}
\begin{Sinput}
> alpha <- fisher.alpha(BCI)
\end{Sinput}
\end{Schunk}

\section{Rarefaction}

Species richness increases with sample size, and differences in
richness actually may be caused by differences in sample size.  To
solve this problem, we may try to rarefy species richness to the same
number of individuals.  Expected number of species in a community
rarefied from $N$ to $n$ individuals is:
\begin{equation}
\label{eq:rare}
\hat S_n = \sum_{i=1}^S (1 - p_i),\, \text{where} \quad p_i = {N-x_i
  \choose n} \Bigm /{N \choose n}
\end{equation}
where $x_i$ is the count of species $i$, and ${N \choose n}$ is the
binomial coefficient, or the number of ways we can choose $n$ from
$N$, and $p_i$ give the probabilities that species $i$ does not occur in a
sample of size $n$.  This is defined only when $N-x_i > n$, but for
other cases $p_i = 0$ or the species is sure to occur in the sample.
The variance of rarefied richness is:
\begin{equation}
\label{eq:rarevar}
s^2 = p_i (1-p_i) + 2 \sum_{i=1}^S \sum_{j>i} \left[ {N- x_i - x_j
    \choose n} \Bigm / {N
    \choose n} - p_i p_j\right]
\end{equation}
Equation \ref{eq:rarevar} actually is of the same form as the variance
of sum of correlated variables:
\begin{equation}
\mathrm{var} \left(\sum x_i \right) = \sum \mathrm{var}(x_i) - 2 \sum_{i=1}^S
\sum_{j>i} \mathrm{cov}(x_i, x_j)
\end{equation}

The number of stems per hectare varies in our
data set:
\begin{Schunk}
\begin{Sinput}
> quantile(rowSums(BCI))
\end{Sinput}
\begin{Soutput}
   0%   25%   50%   75%  100% 
340.0 409.0 428.0 443.5 601.0 
\end{Soutput}
\end{Schunk}
To express richness for the same number of individuals, we can use:
\begin{Schunk}
\begin{Sinput}
> Srar <- rarefy(BCI, min(rowSums(BCI)))
\end{Sinput}
\end{Schunk}
Rarefaction curves often are seen as an objective solution for
comparing species richness with different sample sizes.  However, rank
orders typically differ among different rarefaction sample sizes.

As an extreme case we may rarefy sample size to two individuals:
\begin{Schunk}
\begin{Sinput}
> S2 <- rarefy(BCI, 2)
\end{Sinput}
\end{Schunk}
This will not give equal rank order with the previous rarefaction
richness:
\begin{Schunk}
\begin{Sinput}
> all(rank(Srar) == rank(S2))
\end{Sinput}
\begin{Soutput}
[1] FALSE
\end{Soutput}
\end{Schunk}
Moreover, the rarefied richness for two individuals is a finite
sample variant of Simpson's diversity index (or, more precisely of
$D_1 + 1$), and these two are almost identical in BCI:
\begin{Schunk}
\begin{Sinput}
> range(diversity(BCI, "simp") - (S2 - 1))
\end{Sinput}
\begin{Soutput}
[1] -0.002868298 -0.001330663
\end{Soutput}
\end{Schunk}
Rarefaction is sometimes presented as an ecologically meaningful
alternative to dubious diversity indices, but the differences really
seem to be small.

\section{Taxonomic and functional diversity}

Simple diversity indices only consider species identity: all different
species are equally different. In contrast, taxonomic and functional
diversity indices see how different two different species
are. Taxonomic and functional diversities are used in different fields
of science, but they really have very similar reasoning, and either
could be used either with taxonomic or functional properties of
species.

\subsection{Taxonomic diversity: average distance of properties}

The two basic indices are called taxonomic diversity ($\Delta$) and
taxonomic distinctness ($\Delta^*$):
\begin{align}
  \Delta &= \frac{\sum \sum_{i<j} \omega_{ij} x_i x_j}{n (n-1) / 2}\\
\Delta^* &= \frac{\sum \sum_{i<j} \omega_{ij} x_i x_j}{\sum \sum_{i<j} x_i x_j}
\end{align}
These equations give the index values for a single site, and summation
goes over species $i$ and $j$, and $\omega$ are the taxonomic
distances among taxa, $x$ are species abundances, and $n$ is the total
abundance for a site.  With presence absence data, both indices
reduce to the same index called $\Delta^+$, and for this it is
possible to estimate standard deviation. There are two indices
derived from $\Delta^+$: it can be multiplied with species
richness\footnote{This text normally uses upper case letter $S$ for
  species richness, but lower case $s$ is used here in accordance with
  the original papers on taxonomic diversity}
to give $s \Delta^+$, or it can be used to estimate an index of
variation in taxonomic distinctness $\Lambda^+$:
\begin{equation}
  \Lambda^+ = \frac{\sum \sum_{i<j} \omega_{ij}^2}{n (n-1) / 2} - (\Delta^+)^2
\end{equation}

We still need the taxonomic differences among species ($\omega$) to
calculate the indices. These can be any
distance structure among species, but usually it is found from
established hierarchic taxonomy. Typical coding is that differences
among species in the same genus is $1$, among the same family it is
$2$ etc. However, the taxonomic differences are scaled to maximum
$100$ for easier comparison between different data sets and
taxonomies. Alternatively, it is possible to scale steps between
taxonomic level proportional to the reduction in the number of
categories: if almost all genera have only one species, it does not
make a great difference if two individuals belong to a different
species or to a different genus.

Function \texttt{taxondive} implements indices of taxonomic diversity,
and \texttt{taxa2dist} can be used to convert classification tables to
taxonomic distances either with constant or variable step lengths
between successive categories. There is no taxonomic table for the BCI
data in \texttt{vegan}\footnote{Actually I made such a classification,
  but taxonomic differences proved to be of little use in the Barro
  Colorado data: they only singled out sites with Monocots (palm
  trees) in the data.}
but there is such a table for the Dune meadow data (Fig. \ref{fig:taxondive}):
\begin{Schunk}
\begin{Sinput}
> data(dune)
> data(dune.taxon)
> taxdis <- taxa2dist(dune.taxon, varstep = TRUE)
> mod <- taxondive(dune, taxdis)
\end{Sinput}
\end{Schunk}
\begin{SCfigure}
\includegraphics{diversity-vegan-015}
\caption{Taxonomic diversity $\Delta^+$ for the dune meadow data. The
  points are diversity values of single sites, and the funnel is their
  approximate confidence intervals ($2 \times$ standard error).}
\label{fig:taxondive}
\end{SCfigure}

\subsection{Functional diversity: the height of property tree}

In taxonomic diversity the primary data were taxonomic trees which
were transformed to pairwise distances among species. In functional
diversity the primary data are species properties which are translated
to pairwise distances among species and then to clustering trees of
species properties. The argument for trees is that in this way a
single deviant species will have a small influence, since its
difference is evaluated only once instead of evaluating its distance
to all other species.

Function \texttt{treedive} implements functional diversity defined as
the total branch length in a trait dendrogram connecting all species,
but excluding the unnecessary root segments of the tree.  The example
uses the taxonomic distances of the previous chapter. These are first
converted to a hierarchic clustering (which actually were their
original form before \texttt{taxa2dist} converted them into distances)
\begin{Schunk}
\begin{Sinput}
> tr <- hclust(taxdis, "aver")
> mod <- treedive(dune, tr)
\end{Sinput}
\end{Schunk}

\section{Species abundance models}

Diversity indices may be regarded as variance measures of species
abundance distribution.  We may wish to inspect abundance
distributions more directly.  \texttt{Vegan} has functions for
Fisher's log-series and Preston's log-normal models, and in addition
several models for species abundance distribution.

\subsection{Fisher and Preston}

In Fisher's log-series, the expected number of species $\hat f$ with $n$
individuals is:
\begin{equation}
\hat f_n = \frac{\alpha x^n}{n}
\end{equation}
where $\alpha$ is the diversity parameter, and $x$ is a nuisance
parameter defined by $\alpha$ and total number
of individuals $N$ in the site, $x = N/(N-\alpha)$.  Fisher's
log-series for a randomly selected plot is (Fig. \ref{fig:fisher}):
\begin{Schunk}
\begin{Sinput}
> k <- sample(nrow(BCI), 1)
> fish <- fisherfit(BCI[k, ])
> fish
\end{Sinput}
\begin{Soutput}
Fisher log series model
No. of species: 84 

      Estimate Std. Error
alpha    34.12     4.6967
\end{Soutput}
\end{Schunk}
\begin{SCfigure}
\includegraphics{diversity-vegan-018}
\caption{Fisher's log-series fitted to one randomly selected site
  (12).}
\label{fig:fisher}
\end{SCfigure}
We already saw $\alpha$ as a diversity index.  Now we also obtained
estimate of standard error of $\alpha$ (these also are optionally
available in \texttt{fisherfit}).  The standard errors are based on
the second derivatives (curvature) of log-likelihood at the solution
of $\alpha$.  The distribution of $\alpha$ is often non-normal
and skewed, and standard errors are of not much use.  However,
\texttt{fisherfit} has a \texttt{profile} method that can be used to
inspect the validity of normal assumptions, and will be used in
calculations of confidence intervals from profile deviance:
\begin{Schunk}
\begin{Sinput}
> confint(fish)
\end{Sinput}
\begin{Soutput}
   2.5 %   97.5 % 
25.90110 44.45126 
\end{Soutput}
\end{Schunk}

Preston's log-normal model is the main challenger to Fisher's
log-series.  Instead of plotting species by frequencies, it bins
species into frequency classes of increasing sizes.  As a result,
upper bins with high range of frequencies become more common, and
sometimes the result looks similar to Gaussian distribution truncated
at the left.

There are two alternative functions for the log-normal model:
\texttt{prestonfit} and \texttt{prestondistr}.  Function
\texttt{prestonfit} uses traditionally binning approach, and is burdened
with arbitrary choices of binning limits and treatment of ties.
Function \texttt{prestondistr} directly
maximizes truncated log-normal likelihood without binning data, and it
is the recommended alternative.  Log-normal models  usually fit poorly
to the BCI data, but here our random plot (number 12):
\begin{Schunk}
\begin{Sinput}
> prestondistr(BCI[k, ])
\end{Sinput}
\begin{Soutput}
Preston lognormal model
Method: maximized likelihood to log2 abundances 
No. of species: 84 

     mode     width        S0 
 0.668697  1.776272 22.833365 

Frequencies by Octave
                0        1        2        3        4        5
Observed 34.00000 20.00000 13.00000 6.000000 6.000000 3.000000
Fitted   21.27136 22.43963 17.24209 9.649779 3.933674 1.167973
                 6
Observed 2.0000000
Fitted   0.2525928
\end{Soutput}
\end{Schunk}

\subsection{Ranked abundance distribution}

An alternative approach to species abundance distribution is to plot
logarithmic abundances in decreasing order, or against ranks of
species.  These are known as ranked abundance
distribution curves, species abundance curves, dominance--diversity
curves or Whittaker plots.
Function \texttt{radfit} fits some of the most popular models using
maximum likelihood estimation:
\begin{align}
\hat a_r &= \frac{N}{S} \sum_{k=r}^S \frac{1}{k} &\text{brokenstick}\\
\hat a_r &= N \alpha (1-\alpha)^{r-1} & \text{preemption} \\
\hat a_r &= \exp \left[\log (\mu) + \log (\sigma) \Phi \right]
&\text{log-normal}\\
\hat a_r &= N \hat p_1 r^\gamma &\text{Zipf}\\
\hat a_r &= N c (r + \beta)^\gamma &\text{Zipf--Mandelbrot}
\end{align}
Where $\hat a_r$ is the expected abundance of species at rank $r$, $S$
is the number of species, $N$ is the number of individuals, $\Phi$ is
a standard normal function, $\hat p_1$ is the estimated proportion of
the most abundant species, and $\alpha$, $\mu$, $\sigma$, $\gamma$,
$\beta$ and $c$ are the estimated parameters in each model.

It is customary to define the models for proportions $p_r$ instead of
abundances $a_r$, but there is no reason for this, and \texttt{radfit}
is able to work with the original abundance data.  We have count data,
and the default Poisson error looks appropriate, and our example data
set gives (Fig. \ref{fig:rad}):
\begin{Schunk}
\begin{Sinput}
> rad <- radfit(BCI[k, ])
> rad
\end{Sinput}
\begin{Soutput}
RAD models, family poisson 
No. of species 84, total abundance 366

           par1      par2     par3    Deviance AIC      BIC     
Null                                   97.0832 330.6542 330.6542
Preemption  0.055073                   90.2032 325.7742 328.2050
Lognormal   0.73528   1.239            27.1559 264.7269 269.5885
Zipf        0.16824  -0.91629          17.1743 254.7453 259.6070
Mandelbrot  0.5485   -1.245    2.2134   6.0875 245.6585 252.9509
\end{Soutput}
\end{Schunk}
\begin{SCfigure}
\includegraphics{diversity-vegan-022}
\caption{Ranked abundance distribution models for a random plot
  (no. 12).  The best model has the lowest \textsc{aic}.}
\label{fig:rad}
\end{SCfigure}

Function \texttt{radfit} compares the models using alternatively
Akaike's or Schwartz's Bayesian information criteria.  These are based
on log-likelihood, but penalized by the number of estimated
parameters.  The penalty per parameter is $2$ in \textsc{aic}, and
$\log S$ in \textsc{bic}.  Brokenstick is regarded as a null model and
has no estimated parameters in \texttt{vegan}.  Preemption model has
one estimated parameter ($\alpha$), log-normal and Zipf models two
($\mu, \sigma$, or $\hat p_1, \gamma$, resp.), and Zipf--Mandelbrot
model has three ($c, \beta, \gamma$).

Function \texttt{radfit} also works with data frames, and fits models
for each site. It is curious that log-normal model rarely is the
choice, although it generally is regarded as the canonical model, in
particular in data sets like Barro Colorado tropical forests.

\section{Species accumulation and beta diversity}

Species accumulation models and species pool models study collections
of sites, and their species richness, or try to estimate the number of
unseen species.

\subsection{Species accumulation models}

Species accumulation models are similar to rarefaction: they study the
accumulation of species when the number of sites increases.  There are
several alternative methods, including accumulating sites in the order
they happen to be, and repeated accumulation in random order.  In
addition, there are three analytic models.  Rarefaction pools
individuals together, and applies rarefaction equation (\ref{eq:rare})
to these individuals.  Kindt's exact accumulator resembles rarefaction:
\begin{equation}
\label{eq:kindt}
\hat S_n = \sum_{i=1}^S (1 - p_i), \, \text{where} \quad p_i = {N- f_i
\choose n} \Bigm / {N \choose n}
\end{equation}
where $f_i$ is the frequency of species $i$.  Approximate variance
estimator is:
\begin{equation}
\label{eq:kindtvar}
s^2 = p_i (1 - p_i) + 2 \sum_{i=1}^S \sum_{j>i} \left( r_{ij}
  \sqrt{p_i(1-p_i)} \sqrt{p_j (1-p_j)}\right)
\end{equation}
where $r_{ij}$ is the correlation coefficient between species $i$ and
$j$.  Both of these are unpublished: eq. \ref{eq:kindt} was developed
by Roeland Kindt, and eq. \ref{eq:kindtvar} by Jari Oksanen. The third
analytic method was suggested by Coleman:
\begin{equation}
\label{eq:cole}
S_n = \sum_{i=1}^S (1 - p_i), \, \text{where} \quad p_i = \left(1 - \frac{1}{n}\right)^{f_i}
\end{equation}
and he suggested variance $s^2 = p_i (1-p_i)$ which ignores the
covariance component.  In addition, eq. \ref{eq:cole} does not
properly handle sampling without replacement and underestimates the
species accumulation curve.

The recommended is Kindt's exact method (Fig. \ref{fig:sac}):
\begin{Schunk}
\begin{Sinput}
> sac <- specaccum(BCI)
> plot(sac, ci.type = "polygon", ci.col = "yellow")
\end{Sinput}
\end{Schunk}
\begin{SCfigure}
\includegraphics{diversity-vegan-024}
\caption{Species accumulation curve for the BCI data; exact method.}
\label{fig:sac}
\end{SCfigure}

\subsection{Beta diversity}

Whittaker divided diversity into various components. The best known
are diversity in one spot that he called alpha diversity, and the
diversity along gradients that he called beta diversity. The basic
diversity indices are indices of alpha diversity. Beta diversity
should be studied with respect to gradients, but almost everybody
understand that as a measure of general heterogeneity: how many more
species do you have in a collection of sites compared to an average
site.

The best known index of beta diversity is based on the ratio of total
number of species in a collection of sites ($S$) and the average
richness per one site ($\bar \alpha$):
\begin{equation}
  \label{eq:beta}
  \beta = S/\bar \alpha - 1
\end{equation}
Subtraction of one means that $\beta = 0$ when there are no excess
species or no heterogeneity between sites. For this index, no specific
functions are needed, but this index can be easily found with the help
of \texttt{vegan} function \texttt{specnumber}:
\begin{Schunk}
\begin{Sinput}
> ncol(BCI)/mean(specnumber(BCI)) - 1
\end{Sinput}
\begin{Soutput}
[1] 1.478519
\end{Soutput}
\end{Schunk}

The index of eq. \ref{eq:beta} is problematic because $S$ increases
with the number of sites even when sites are all subsets of the same
community.  Whittaker noticed this, and suggested the index to be
found from pairwise comparison of sites. If the number of shared
species in two sites is $a$, and the numbers of species unique to each
site are $b$ and $c$, then $\bar \alpha = (2a + b + c)/2$ and $S =
a+b+c$, and index \ref{eq:beta} can be expressed as:
\begin{equation}
  \label{eq:betabray}
  \beta = \frac{a+b+c}{(2a+b+c)/2} - 1 = \frac{b+c}{2a+b+c}
\end{equation}
This is the S{\o}rensen index of dissimilarity, and it can be found
for all sites using \texttt{vegan} function \texttt{vegdist} with
binary data:
\begin{Schunk}
\begin{Sinput}
> beta <- vegdist(BCI, binary = TRUE)
> mean(beta)
\end{Sinput}
\begin{Soutput}
[1] 0.3399075
\end{Soutput}
\end{Schunk}

There are many other definitions of beta diversity in addition to
eq. \ref{eq:beta}.  All commonly used indices can be found using
\texttt{betadiver}. The indices in \texttt{betadiver} can be referred
to by subscript name, or index number:
\begin{Schunk}
\begin{Sinput}
> betadiver(help = TRUE)
\end{Sinput}
\begin{Soutput}
1 "w" = (b+c)/(2*a+b+c)
2 "-1" = (b+c)/(2*a+b+c)
3 "c" = (b+c)/2
4 "wb" = b+c
5 "r" = 2*b*c/((a+b+c)^2-2*b*c)
6 "I" = log(2*a+b+c)-2*a*log(2)/(2*a+b+c)-((a+b)*log(a+b)+(a+c)*log(a+c))/(2*a+b+c)
7 "e" = exp(log(2*a+b+c)-2*a*log(2)/(2*a+b+c)-((a+b)*log(a+b)+(a+c)*log(a+c))/(2*a+b+c))-1
8 "t" = (b+c)/(2*a+b+c)
9 "me" = (b+c)/(2*a+b+c)
10 "j" = a/(a+b+c)
11 "sor" = 2*a/(2*a+b+c)
12 "m" = (2*a+b+c)*(b+c)/(a+b+c)
13 "-2" = pmin(b,c)/(pmax(b,c)+a)
14 "co" = (a*c+a*b+2*b*c)/(2*(a+b)*(a+c))
15 "cc" = (b+c)/(a+b+c)
16 "g" = (b+c)/(a+b+c)
17 "-3" = pmin(b,c)/(a+b+c)
18 "l" = (b+c)/2
19 "19" = 2*(b*c+1)/((a+b+c)^2+(a+b+c))
20 "hk" = (b+c)/(2*a+b+c)
21 "rlb" = a/(a+c)
22 "sim" = pmin(b,c)/(pmin(b,c)+a)
23 "gl" = 2*abs(b-c)/(2*a+b+c)
24 "z" = (log(2)-log(2*a+b+c)+log(a+b+c))/log(2)
\end{Soutput}
\end{Schunk}
Some of these indices are duplicates, and many of them are well known
dissimilarity indices.
One of the more interesting indices is based
on the Arrhenius species--area model
\begin{equation}
  \label{eq:arrhenius}
  \hat S = c X^z
\end{equation}
where $X$ is the area (size) of the patch or site, and $c$ and $z$ are
parameters. Parameter $c$ is uninteresting, but $z$ gives the
steepness of the species area curve and is a measure of beta
diversity. In islands,  $z$ is typically about $0.3$. This kind of
islands can be regarded as subsets of the same community, indicating
that we really should talk about gradient differences if $z > 0.3$. We
can find the value of $z$ for a pair of plots using function
\texttt{betadiver}:
\begin{Schunk}
\begin{Sinput}
> z <- betadiver(BCI, "z")
> quantile(z)
\end{Sinput}
\begin{Soutput}
       0%       25%       50%       75%      100% 
0.2732845 0.3895024 0.4191536 0.4537180 0.5906091 
\end{Soutput}
\end{Schunk}
The size $X$ and parameter $c$ cancel out, and the index gives the
estimate $z$ for any pair of sites.

Function \texttt{betadisper} can be used to analyse beta diversities
with respect to classes or factors.  There is no such classification
available for the Barro Colorado Island data, and the example studies
beta diversities in the management classes of the dune meadows
(Fig. \ref{fig:betadisper}):
\begin{Schunk}
\begin{Sinput}
> data(dune)
> data(dune.env)
> z <- betadiver(dune, "z")
> mod <- with(dune.env, betadisper(z, Management))
> mod
\end{Sinput}
\begin{Soutput}
	Homogeneity of multivariate dispersions

Call: betadisper(d = z, group = Management)

No. of Positive Eigenvalues: 12
No. of Negative Eigenvalues: 7

Average distance to centroid:
    BF     HF     NM     SF 
0.2600 0.2570 0.4419 0.3667 

Eigenvalues for PCoA axes:
  PCoA1   PCoA2   PCoA3   PCoA4   PCoA5   PCoA6   PCoA7   PCoA8   PCoA9 
 1.6547  0.8870  0.5334  0.3744  0.2873  0.2245  0.1613  0.0810  0.0652 
 PCoA10  PCoA11  PCoA12  PCoA13  PCoA14  PCoA15  PCoA16  PCoA17  PCoA18 
 0.0353  0.0183  0.0040 -0.0042 -0.0194 -0.0369 -0.0429 -0.0536 -0.0602 
 PCoA19 
-0.0828 
\end{Soutput}
\end{Schunk}
\begin{SCfigure}
\includegraphics{diversity-vegan-030}
\caption{Box plots of beta diversity measured as the average steepness
  ($z$) of the species area curve in the Arrhenius model $S = cX^z$ in
  Management classes of dune meadows.}
\label{fig:betadisper}
\end{SCfigure}

\section{Species pool}
\subsection{Number of unseen species}

Species accumulation models indicate that not all species were seen in
any site.  These unseen species also belong to the species pool.
Functions \texttt{specpool} and \texttt{estimateR} implement some
methods of estimating the number of unseen species.  Function
\texttt{specpool} studies a collection of sites, and
\texttt{estimateR} works with counts of individuals, and can be used
with a single site.  Both functions assume that the number of unseen
species is related to the number of rare species, or species seen only
once or twice.

Function \texttt{specpool} implements the following models to estimate
the pool size $S_p$:
\begin{align}
S_p &= S_o + \frac{f_1^2}{2 f_2} & \text{Chao}\\
S_p &= S_o + f_1 \frac{N-1}{N} & \text{1st order Jackknife}\\
S_p & = S_o + f_1 \frac{2N-3}{N} + f_2 \frac{(N-2)^2}{N(N-1)} &
\text{2nd order Jackknife}\\
S_p &= S_o + \sum_{i=1}^{S_o} (1-p_i)^N & \text{Bootstrap}
\end{align}
Here $S_o$ is the observed number of species, $f_1$ and $f_2$ are the
numbers of species observed once or twice, $N$ is the number of sites,
and $p_i$ are proportions of species.  The idea in jackknife seems to
be that we missed about as many species as we saw only once, and the
idea in bootstrap that if we repeat sampling (with replacement) from
the same data, we miss any many species as we missed originally.

The variance estimators of Chao is:
\begin{equation}
s^2 = f_2 \left(\frac{G^4}{4} + G^3 + \frac{G^2}{2} \right), \,
\text{where}\quad G = \frac{f_1}{f_2}
\end{equation}
The variance of the first-order jackknife is based on the number of
``singletons'' $r$ (species occurring only once in the data) in sample
plots:
\begin{equation}
s^2 = \left(\sum_{i=1}^N r_i^2 - \frac{f_1}{N}\right) \frac{N-1}{N}
\end{equation}
Variance of the second-order jackknife is not evaluated in
\texttt{specpool} (but contributions are welcome).
For the variance of bootstrap estimator, it is practical to define a
new variable $q_i = (1-p_i)^N$ for each species:
\begin{equation}
\begin{split}
s^2 = \sum_{i=1}^{S_o} q_i (1-q_i) + 2 \sum \sum Z_p , \quad \text{where} \\
Z_p = \dots
\end{split}
\end{equation}

The extrapolated richness values for the whole BCI data are:
\begin{Schunk}
\begin{Sinput}
> specpool(BCI)
\end{Sinput}
\begin{Soutput}
    Species     chao  chao.se  jack1 jack1.se    jack2     boot
All     225 236.6053 6.659395 245.58 5.650522 247.8722 235.6862
     boot.se  n
All 3.468888 50
\end{Soutput}
\end{Schunk}
If the estimation of pool size really works, we should get the same
values of estimated richness if we take a random subset of a half of
the plots:
\begin{Schunk}
\begin{Sinput}
> s <- sample(nrow(BCI), 25)
> specpool(BCI[s, ])
\end{Sinput}
\begin{Soutput}
    Species     chao  chao.se  jack1 jack1.se    jack2     boot
All     212 251.3846 18.77757 242.72  8.69317 260.6983 225.6209
     boot.se  n
All 4.709401 25
\end{Soutput}
\end{Schunk}
These typically are even lower than the observed richness
(225 species) at the whole data set.

\subsection{Pool size from a single site}

The \texttt{specpool} function needs a collection of sites, but there
are some methods that estimate the number of unseen species for each
single site.  These functions need counts of individuals, and species
seen only once or twice, or other rare species, take the place of
species with low frequencies.  Function \texttt{estimateR} implements
two of these methods:
\begin{Schunk}
\begin{Sinput}
> estimateR(BCI[k, ])
\end{Sinput}
\begin{Soutput}
                 12
S.obs     84.000000
S.chao1  110.714286
se.chao1  12.996942
S.ACE    128.001262
se.ACE     6.823939
\end{Soutput}
\end{Schunk}
Chao's method is similar as above, but uses another, ``unbiased''
equation. \textsc{ace} is based on rare species also:
\begin{equation}
\begin{split}
S_p &= S_\mathrm{abund} + \frac{S_\mathrm{rare}}{C_\mathrm{ACE}} +
\frac{a_1}{C_\mathrm{ACE}} \gamma^2 \quad \text{where}\\
C_\mathrm{ACE} &= 1 - \frac{a_1}{N_\mathrm{rare}}\\
\gamma^2 &= \frac{S_\mathrm{rare}}{C_\mathrm{ACE}} \sum_{i=1}^{10} i
(i-1) a_1 \frac{N_\mathrm{rare} - 1}{N_\mathrm{rare}}
\end{split}
\end{equation}
Now $a_1$ takes the place of $f_1$ above, and means the number of
species with only one individual.
Here $S_\mathrm{abund}$ and $S_\mathrm{rare}$ are the numbers of
species of abundant and rare species, with an arbitrary upper limit of
10 individuals for a rare species, and $N_\mathrm{rare}$ is the total
number of individuals in rare species.

The pool size
is estimated separately for each site, but if input is a data frame,
each site will be analysed.

If log-normal abundance model is appropriate, it can be used to
estimate the pool size.  Log-normal model has a finite number of
species which can be found integrating the log-normal:
\begin{equation}
S_p = S_\mu \sigma \sqrt{2 \pi}
\end{equation}
where $S_\mu$ is the modal height or the expected number of species at
maximum (at $\mu$), and $\sigma$ is the width.  Function
\texttt{veiledspec} estimates this integral from a model fitted either
with \texttt{prestondistr} or \texttt{prestonfit}, and fits the latter
if raw site data are given.  Log-normal model fits badly, and
\texttt{prestonfit} is particularly poor.  Therefore the following
explicitly uses \texttt{prestondistr}, although this also may fail:
\begin{Schunk}
\begin{Sinput}
> veiledspec(prestondistr(BCI[k, ]))
\end{Sinput}
\begin{Soutput}
Extrapolated     Observed       Veiled 
   101.66452     84.00000     17.66452 
\end{Soutput}
\begin{Sinput}
> veiledspec(BCI[k, ])
\end{Sinput}
\begin{Soutput}
Extrapolated     Observed       Veiled 
          NA           84           NA 
\end{Soutput}
\end{Schunk}

\subsection{Probability of pool membership}

Beals smoothing was originally suggested as a tool of regularizing data
for ordination.  It regularizes data too strongly,
but it has been suggested as a method of estimating which of the
missing species could occur in a site, or which sites are suitable for
a species.  The probability for each species at each site is assessed
from other species occurring on the site.

Function \texttt{beals} implement Beals smoothing:
\begin{Schunk}
\begin{Sinput}
> smo <- beals(BCI)
\end{Sinput}
\end{Schunk}
We may see how the estimated probability of occurrence and observed
numbers of stems relate in one of the more familiar species. We study
only one species, and to avoid circular reasoning we do not include
the target species in the smoothing (Fig. \ref{fig:beals}):
\begin{Schunk}
\begin{Sinput}
> j <- which(colnames(BCI) == "Ceiba.pentandra")
> plot(beals(BCI, species = j, include = FALSE), BCI[, j], 
+     main = "Ceiba pentandra", xlab = "Probability of occurrence", 
+     ylab = "Occurrence")
\end{Sinput}
\end{Schunk}
\begin{SCfigure}
\includegraphics{diversity-vegan-037}
\caption{Beals smoothing for \emph{Ceiba pentandra}.}
\label{fig:beals}
\end{SCfigure}

\end{document}
